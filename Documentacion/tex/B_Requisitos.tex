\apendice{Especificación de Requisitos}

\section{Introducción}

El objetivo general de este programa es la depuración de código en lenguaje C. Para ello se requerirá poder abrir, crear, editar, compilar y depurar archivos con la extensión .c

\section{Objetivos generales}

\item OBJ-01:Depuración de código C.

\section{Catalogo de requisitos}

\item RI-01:Abrir archivos .c y .h
\item RI-02:Crear archivos .c y .h
\item RI-03:Editar archivos .c y .h
\item RI-04:Guardar archivos .c y .h
\item RI-05:Compilación de archivos .c
\item RI-06:Depuración de archivos .c

\section{Especificación de requisitos}


\begin{table}[h!]
\centering
\begin{tabular}[c]{|c|m{10cm}|}
\hline
RF-01 & Abrir archivos .c y .h \\
\hline
\hline
Versión & Versión 1.0 \\
\hline
Autor & Ruben Marcos Gonzalez \\
\hline
Tutor & Carlos Pardo Aguilar \\
\hline
Objetivos asociados & OBJ-01:Depuración de código C. \\
\hline
Requisitos asociados & RI-01:Abrir archivos .c y .h \\
\hline
Descripción & El programa deberá ser capaz de abrir archivos con la extensión .c y .h \\
\hline
Precondición & El usuario deberá haber iniciado el asistente \\
\hline
\multirow{2}{*}{Secuencia normal} & Paso 1: solicita al sistema el explorador de archivos \\
\hline
& Paso 2: una vez seleccionado el archivo se accederá al editor \\
\hline
\multirow{2}{*}{Excepciones} & Paso 2: el usuario abrirá un archivo con extensión no permitida \\
\hline
& Paso 2: el usuario abrirá un archivo corrupto \\
\hline
Postcondicón & El usuario deberá tener cargado un archivo con la extensión .c o .h \\
\hline
\end{tabular}
\end{table}

\begin{table}[h!]
\centering
\begin{tabular}[c]{|c|m{10cm}|}
\hline
RF-04 & Abrir archivos .c y .h \\
\hline
\hline
Versión & Versión 1.0 \\
\hline
Autor & Ruben Marcos Gonzalez \\
\hline
Tutor & Carlos Pardo Aguilar \\
\hline
Objetivos asociados & OBJ-01:Depuración de código C. \\
\hline
Requisitos asociados & RI-04:Guardar archivos .c y .h \\
\hline
Descripción & El programa deberá ser capaz de guardar archivos con la extensión .c y .h \\
\hline
Precondición & El usuario deberá tener cargado en el programa un archivo \\
\hline
\multirow{2}{*}{Secuencia normal} & Paso 1: solicita al sistema el explorador de archivos \\
\hline
& Paso 2: se guardara el archivo con el nombre y la extensión elegidas \\
\hline
\multirow{2}{*}{Excepciones} & Paso 2: no haya espacio en el disco duro, se vuelve al paso 1 \\
\hline
& Paso 2: existe un archivo con el mismo nombre, se solicitara al usuario si quiere sobrescribirlo \\
\hline
Postcondicón & El usuario deberá haber cargado un archivo con la extensión .c o .h \\
\hline
\end{tabular}
\end{table}

\begin{table}[h!]
\centering
\begin{tabular}[c]{|c|m{10cm}|}
\hline
RF-05 & Compilar archivos .c \\
\hline
\hline
Versión & Versión 1.0 \\
\hline
Autor & Ruben Marcos Gonzalez \\
\hline
Tutor & Carlos Pardo Aguilar \\
\hline
Objetivos asociados & OBJ-01:Depuración de código C. \\
\hline
Requisitos asociados & RI-05:Compilar archivos .c \\
\hline
Descripción & El programa deberá ser capaz de compilar archivos .c y crear ejecutables a partir de ellos \\
\hline
Precondición & El usuario deberá haber guardado los cambios del archivo \\
\hline
\multirow{2}{*}{Secuencia normal} & Paso 1: creará el archivo .o correspondiente a dicho archivo \\
\hline
& Paso 2: generará un ejecutable a partir del archivo .c actual \\
\hline
\multirow{2}{*}{Excepciones} & Paso 1: el código este mal escrito lo que genera un error de compilación y la interrupción de la misma \\
\hline
& Paso 2: el archivo modificado no ha sido guardado por lo que se ejecutara el requisito RI-04 \\
\hline
Postcondicón & -- \\
\hline
\end{tabular}
\end{table}
