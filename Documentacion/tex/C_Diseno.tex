\apendice{Especificación de diseño}

\section{Introducción}

Aquí se especificará las estructuras de datos utilizadas al igual que las trazas de las funciones empleadas.

\section{Diseño de datos}

Esta aplicación no guarda datos en ningún sitio más allá de los archivos C creados por el usuario,
esto no quiere decir que no tenga estructuras de datos internas con las que trabaja.\\
Esta aplicación hay 5 estructuras de datos grandes\footnote{En el caso de usar bicolas o listas es por optimización, en caso de que las operaciones mas utilizadas sean las de indexación se usan listas, y si en caso contrario las operaciones mas usadas sean las de push(llamada append en Python) y pop se usaran bicolas}: \begin{enumerate}
\item funciones: Es un diccionario que guarda las funciones declaradas en el archivo sobre el que estás trabajando, la clave del diccionario es el nombre de la función y el valor es una bicola que cada elemento es una linea de la función
\item iterexec: es una lista de bicolas, las bicolas son las funciones que se van a ejecutar y se guardan en esta lista en el orden que se deben ejecutar, en caso de que se invoque a una función se añade a esta lista y en caso de un return se borra
\item vardicts: es una lista de diccionarios, en dichos diccionarios se almacenan las variables que se usan en cada función, la clave de estos diccionarios es el nombre de la variable y el valor es una lista compuesta por el tipo de la variable y el valor de la misma. Cada vez que se entra a una función se genera un nuevo diccionario y cada vez que se sale, bien por un return o porque se ha llegado al final de la función, se borra uno
\item estructuras: es un diccionario que como clave tiene las estructuras que son utilizadas en el archivo sobre el que estamos trabajando, la clave es el nombre de la estructura y el valor es un diccionario que actua de forma similar a los usados en vardicts que se componen por el nombre de los campos por claves y su tipo y valor como los valores del diccionario
\item retornos: es una bicola para manejar los retornos del archivo que estamos ejecutando, cada vez que se entra una función se añade un elemento a la bicola y si se llega un retorno se extrae un elemento para ser modificado
\end{enumerate}

\section{Diseño procedimental}

Lo primero de todo para poder hacer funcionar la aplicación debemos trabajar sobre un archivo, para ello tenemos dos opciones crear uno nuevo (ver traza:\ref{fig:flow_new}) y guardarlo o cargar (ver traza:\ref{fig:flow_open}) uno ya creado, una vez guardado (ver traza:\ref{fig:flow_save}) el archivo ya se podría compilar. Si compilamos (ver traza:\ref{fig:flow_compile}) el archivo podremos acceder a las opciones de ejecutar (ver traza:\ref{fig:flow_exe}) y activar el modo debugger (ver traza:\ref{fig:flow_bug}). Si accedemos al modo debugger podemos optar por ejecutar una linea entrando (ver traza:\ref{fig:flow_linea1}) o no (ver traza:\ref{fig:flow_linea2}) en las funciones.\\Para ver una traza general del programa ver:\ref{fig:flow_gen}\\
\begin{figure}
    \begin{tikzpicture}[node distance=2cm]
    \node (start) [startstop] {Inicio};
    \node (open) [process, below of=start] {Abrir archivo};
    \node (new) [process, right of=open, xshift= 2cm] {Nuevo archivo};
    \node (edit) [io, below of=open] {Editar archivo};
    \node (save) [process, below of=edit] {Guardar archivo};
    \node (comp) [process, below of=save] {Compilar};
    \node (bug) [process, below of=comp] {Debuggear};
    \node (exe) [process, below of=comp, xshift= 7cm] {Ejecutar};
    \node (linea1) [process, below of=bug, yshift=-1cm] {Ejecutar 1 linea entrando en función};
    \node (linea2) [process, right of=linea1, xshift= 2cm] {Ejecutar 1 linea saltando función};
    \node (left) [decision, below of=linea1, yshift= -1.5cm] {¿Quedan lineas por ejecutar?};
    \node (continue) [decision, left of=left, xshift= -3cm] {¿Seguir debuggeo?};
    \node (stop) [startstop, below of=left, yshift= -1.5cm] {Fin};
    
    \draw [arrow] (start) -- (open);
    \draw [arrow] (start) -| (new);
    \draw [arrow] (open) -- (edit);
    \draw [arrow] (new) |- (save);
    \draw [arrow] (edit) -- (save);
    \draw [arrow] (save) -- (comp);
    \draw [arrow] (comp) -- (bug);
    \draw [arrow] (comp) -| (exe);
    \draw [arrow] (bug) -- (linea1);
    \draw [arrow] (bug) -| (linea2);
    \draw [arrow] (left) -- node[anchor=east] {no} (stop);
    \draw [arrow] (left) -- node[anchor=north] {si} (continue);
    \draw [arrow] (continue) |- node[anchor=south] {si} (bug);
    \draw [arrow] (continue) |- node[anchor=north] {no} (stop);
    \draw [arrow] (linea1) -- (left);
    \draw [arrow] (linea2) |- (left);
    \draw [arrow] (exe) |- (stop);
    \end{tikzpicture}
    \caption{Ejemplo de una ejecución normal del programa}
    \label{fig:flow_gen}
\end{figure}

\begin{figure}
    \centering
    \begin{tikzpicture}[node distance=2cm]
    \node (start) [startstop] {Inicio};
    \node (code) [process, below of=start] {Cambiar el código a lo básico};
    \node (name) [process, below of=code, yshift= -1.5cm] {Borrar la variable donde se almacena el nombre del archivo sobre el que trabajamos};
    \node (stop) [startstop, below of=name, yshift= -1cm] {Fin};
    
    \draw [arrow] (start) -- (code);
    \draw [arrow] (code) -- (name);
    \draw [arrow] (name) -- (stop);
    \end{tikzpicture}
    \caption{Traza de la función nuevo}
    \label{fig:flow_new}
\end{figure}

\begin{figure}
    \centering
    \begin{tikzpicture}[node distance=2cm]
    \node (start) [startstop] {Inicio};
    \node (name) [io, below of=start] {Nombre del archivo};
    \node (exist) [decision, below of=name, yshift= -1.5cm] {¿existe el archivo?};
    \node (inex) [startstop, right of=exist, xshift= 3cm] {No existe archivo fin};
    \node (ext) [decision, below of=exist, yshift= -3cm] {¿Tiene extensión .c?};
    \node (errext) [startstop, left of=ext,xshift = -3.2cm] {Extensión incorrecta fin};
    \node (code) [process, below of=ext, yshift= -2cm] {copia contenido del archivo en el cuadro de código};
    \node (save) [process, below of=code, yshift= -1.2cm] {Almacena el nombre del archivo sobre el que trabajamos};
    \node (stop) [startstop, below of=save, yshift= -0.8cm] {Fin};
    
    \draw [arrow] (start) -- (name);
    \draw [arrow] (name) -- (exist);
    \draw [arrow] (exist) -- node[anchor=north] {no} (inex);
    \draw [arrow] (exist) -- node[anchor=east] {si} (ext);
    \draw [arrow] (ext) -- node[anchor=north] {no} (errext);
    \draw [arrow] (ext) -- node[anchor=east] {si} (code);
    \draw [arrow] (code) -- (save);
    \draw [arrow] (save) -- (stop);
    \end{tikzpicture}
    \caption{Traza de la función abrir}
    \label{fig:flow_open}
\end{figure}

\begin{figure}
    \centering
    \begin{tikzpicture}[node distance=2cm]
    \node (start) [startstop] {Inicio};
    \node (name) [io, below of=start] {Nombre del archivo};
    \node (ext) [decision, below of=name, yshift= -1.5cm] {¿Tiene extensión .c?};
    \node (errext) [startstop, left of=ext,xshift = -3.2cm] {Extensión incorrecta fin};
    \node (exist) [decision, below of=ext, yshift= -3cm] {¿existe el archivo?};
    \node (inex) [process, right of=exist, xshift= 3cm] {Crea nuevo archivo};
    \node (code) [process, below of=exist, yshift= -2cm] {sobreescribe el codigo del archivo};
    \node (stop) [startstop, below of=code, yshift= -0.8cm] {Fin};
    
    \draw [arrow] (start) -- (name);
    \draw [arrow] (name) -- (ext);
    \draw [arrow] (exist) -- node[anchor=north] {no} (inex);
    \draw [arrow] (exist) -- node[anchor=east] {si} (code);
    \draw [arrow] (ext) -- node[anchor=north] {no} (errext);
    \draw [arrow] (ext) -- node[anchor=east] {si} (exist);
    \draw [arrow] (inex) |- (stop);
    \draw [arrow] (code) -- (stop);
    \end{tikzpicture}
    \caption{Traza de la función guardar}
    \label{fig:flow_save}
\end{figure}

\begin{figure}
    \centering
    \begin{tikzpicture}[node distance=2cm]
    \node (start) [startstop] {Inicio};
    \node (saved) [decision, below of=name, yshift= -1.5cm] {¿se han guardado los cambios?};
    \node (save) [process, left of=saved, xshift = -3.2cm] {Se llama a guardar};
    \node (comp) [process, below of=saved, yshift= -2.5cm] {Se compila el archivo};
    \node (stop) [startstop, below of=code] {Fin};
    
    \draw [arrow] (start) -- (saved);
    \draw [arrow] (saved) -- node[anchor=north] {no} (save);
    \draw [arrow] (saved) -- node[anchor=east] {si} (comp);
    \draw [arrow] (save) |- (comp);
    \draw [arrow] (comp) -- (stop);
    \end{tikzpicture}
    \caption{Traza de la función compilar}
    \label{fig:flow_compile}
\end{figure}

\begin{figure}
    \centering
    \begin{tikzpicture}[node distance=2cm]
    \node (start) [startstop] {Inicio};
    \node (save1) [decision, below of=name] {¿se han guardado los cambios?};
    \node (save2) [process, left of=save1, xshift = -3.2cm] {Se llama a guardar};
    \node (comp1) [decision, below of=save1, yshift= -3.3cm] {¿se ha compilado el archivo?};
    \node (comp2) [process, right of=comp1, xshift = 3.2cm] {Se llama a compilar};
    \node (exe) [process, below of=comp1, yshift= -2.5cm] {Se llama al sistema para que ejecute el programa};
    \node (stop) [startstop, below of=code] {Fin};
    
    \draw [arrow] (start) -- (save1);
    \draw [arrow] (save1) -- node[anchor=north] {no} (save2);
    \draw [arrow] (save1) -- node[anchor=east] {si} (comp1);
    \draw [arrow] (comp1) -- node[anchor=north] {no} (comp2);
    \draw [arrow] (comp1) -- node[anchor=east] {si} (exe);
    \draw [arrow] (save2) |- (comp1);
    \draw [arrow] (comp2) |- (exe);
    \draw [arrow] (exe) -- (stop);
    \end{tikzpicture}
    \caption{Traza de la función compilar}
    \label{fig:flow_exe}
\end{figure}

\begin{figure}
    \centering
    \begin{tikzpicture}[node distance=2cm]
    \node (start) [startstop] {Inicio};
    \node (save1) [decision, below of=name] {¿se han guardado los cambios?};
    \node (save2) [process, left of=save1, xshift = -3.2cm] {Se llama a guardar};
    \node (comp1) [decision, below of=save1, yshift= -3.3cm] {¿se ha compilado el archivo?};
    \node (comp2) [process, right of=comp1, xshift = 3.2cm] {Se llama a compilar};
    \node (bug1) [process, below of=comp1, yshift= -2.5cm] {Se transforma el código C original en un árbol iterable};
    \node (bug2) [process, below of=bug1, yshift= -1.5cm] {Se crean las estructuras de datos con las que va a trabajar el depurador};
    \node (stop) [startstop, below of=bug2, yshift= -1cm] {Fin};
    
    \draw [arrow] (start) -- (save1);
    \draw [arrow] (save1) -- node[anchor=north] {no} (save2);
    \draw [arrow] (save1) -- node[anchor=east] {si} (comp1);
    \draw [arrow] (comp1) -- node[anchor=north] {no} (comp2);
    \draw [arrow] (comp1) -- node[anchor=east] {si} (exe);
    \draw [arrow] (save2) |- (comp1);
    \draw [arrow] (comp2) |- (bug1);
    \draw [arrow] (bug1) -- (bug2);
    \draw [arrow] (bug2) -- (stop);
    \end{tikzpicture}
    \caption{Traza de la función modo debug}
    \label{fig:flow_bug}
\end{figure}

\begin{figure}
    \centering
    \begin{tikzpicture}[node distance=2cm]
    \node (start) [startstop] {Inicio};
    \node (linea) [process, below of=start] {Ejecuta siguente linea};
    \node (sub) [process, below of=linea,] {Marca la linea ejecutada};
    \node (var1) [process, below of=sub, yshift= -1cm] {Modifica las variables acorde a la linea ejecutada};
    \node (var2) [io, below of=var1, yshift= -1cm] {Muestra el valor de las variables por pantalla};
    \node (cons) [io, below of=var2, yshift= -1cm] {Escribe por la consola lo que exija la linea};
    \node (stop) [startstop, below of=cons, yshift= -1cm] {Fin};
    
    \draw [arrow] (start) -- (linea);
    \draw [arrow] (linea) -- (sub);
    \draw [arrow] (sub) -- (var1);
    \draw [arrow] (var1) -- (var2);
    \draw [arrow] (var2) -- (cons);
    \draw [arrow] (cons) -- (stop);
    \end{tikzpicture}
    \caption{Traza de la función ejecutar una linea entrando en función}
    \label{fig:flow_linea1}
\end{figure}

\begin{figure}
    \centering
    \begin{tikzpicture}[node distance=2cm]
    \node (start) [startstop] {Inicio};
    \node (linea1) [process, below of=start] {Ejecuta siguente linea};
    \node (linea2) [decision, below of=linea1, yshift= -1.5cm] {¿Entra en otra función?};
    \node (var1) [process, right of=linea2, xshift= 3cm] {Modifica las variables acorde a la linea ejecutada};
    \node (var2) [process, below of=linea2, yshift=-2cm] {Modifica las variables acorde a la linea ejecutada};
    \node (linea3) [process, right of=var1, xshift= 3cm] {Ejecuta siguiente linea};
    \node (linea4) [decision, below of=var1, yshift= -2cm] {¿sale de la función?};
    \node (var3) [io, below of=linea2, yshift= -5cm] {Muestra el valor de las variables por pantalla};
    \node (cons) [io, below of=var3, yshift= -1cm] {Escribe por la consola lo que exija la linea};
    \node (stop) [startstop, below of=cons, yshift= -1cm] {Fin};
    
    \draw [arrow] (start) -- (linea1);
    \draw [arrow] (linea1) -- (linea2);
    \draw [arrow] (linea2) -- node[anchor=north] {si} (var1);
    \draw [arrow] (var1) -- (linea3);
    \draw [arrow] (linea3) |- (linea4);
    \draw [arrow] (linea4) -- node[anchor=east] {no} (var1);
    \draw [arrow] (linea2) -- node[anchor=east] {no} (var2);
    \draw [arrow] (linea4) -- node[anchor=north] {si} (var2);
    \draw [arrow] (var2) -- (var3);
    \draw [arrow] (var3) -- (cons);
    \draw [arrow] (cons) -- (stop);
    \end{tikzpicture}
    \caption{Traza de la función ejecutar una linea saltando función}
    \label{fig:flow_linea2}
\end{figure}

\clearpage

\section{Diseño arquitectónico}

En cuanto al numero de usuarios que van a usar la aplicación  y el número de funciones a las que van a poder acceder es muy simple. Habrá un solo usuario llamado alumno el cual tendrá acceso a todas las funciones de la aplicación. Ver\ref{fig:diagrama de usuarios}

En cuanto a como interactúan las estructuras de datos con las funciones tampoco es complejo ya que las funciones externas ajenas a la depuración no modifican dichas estructuras. Las funciones propias de la depuración interactúan de la siguiente forma:
\begin{itemize}
    \item El primer paso es llamar al modo debug el cual crea las estructuras de: funciones, iterexec, vardicts, estructuras y retornos. Las que primero se crean son funciones y estructuras, posteriormente se añade a iterexec el main y a vardicts un diccionario vacío donde el main almacenará las variables y finalmente se crea retornos que se inicializara como una bicola vacía.
    \item Cada vez que ejecutar linea llega a la inicialización de una variable esta se añadirá al diccionario correspondiente a la función que actualmente estamos depurando con valor desconocido si no se inicializa con ningún valor o con el valor con el que se inicialice si se da el caso.
    \item Cada vez que se declara una variable como un struct se busca dicho struct en el diccionario de estructuras y se añade la variable con el tipo struct al diccionario correspondiente de vardicts a la función actual.
    \item cada vez que se altera una variable se busca en el diccionario correspondiente a la función que se está ejecutando actualmente y se modifica su valor.
    \item Cada vez que se llama a una función se busca en funciones la misma y se añade a iterexec para ser ejecutada, también se añade a vardicts un diccionario para las variables de la función recién añadida y en caso de que la función no este declarada como void se añadirá a la pila de retornos un retorno.
    \item Cada vez que se llega a un retorno se devuelve un valor de la pila de retornos y se vacía la pila de la ejecución de la función que se está depurando.
\end{itemize}

\clearpage

\begin{figure}
    \begin{tikzpicture}[node distance=2cm]
    \node (alu) [startstop] {Usuario: Alumno};
    \node (new) [process, right of=alu, xshift= 5cm, yshift=7cm] {Función: nuevo};
    \node (open) [process, below of=new] {Función: abrir};
    \node (save) [process, below of=open] {Función: guardar};
    \node (comp) [process, below of=save] {Función: compilar};
    \node (bug) [process, below of=comp] {Función: modo debug};
    \node (linea1) [process, below of=bug] {Función: ejecutar linea entrando en función};
    \node (linea2) [process, below of=linea1, yshift=-1cm] {Función: ejecutar linea saltando función};
    
    \draw [arrow] (alu) -- (new);
    \draw [arrow] (alu) -- (save);
    \draw [arrow] (alu) -- (open);
    \draw [arrow] (alu) -- (comp);
    \draw [arrow] (alu) -- (bug);
    \draw [arrow] (alu) -- (linea1);
    \draw [arrow] (alu) -- (linea2);
    \end{tikzpicture}
    \caption{Diagrama de usuarios}
    \label{fig:diagrama de usuarios}
\end{figure}