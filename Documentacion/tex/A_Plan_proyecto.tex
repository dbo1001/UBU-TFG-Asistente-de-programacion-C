\apendice{Plan de Proyecto Software}

\section{Introducción}

Se quiere crear un asistente de programación para C y en este anexo se comentará acerca del planificación que se ha seguido y su viabilidad

\section{Planificación temporal}

El proyecto se irá planificando semana a semana, estableciendo tareas y horario.
Si una tarea no se ha acabado en una semana se aplazará a la siguiente semana o se aplazará indefinidamente si han surgido muchas contrariedades en la realización de la misma dejando la opción de empezar con otras tareas.

\section{Estudio de viabilidad}

El proyecto en primera instancia pare plausible tanto técnicamente como legalmente pero no se sabe hasta que punto las diferencias que hay entre el lenguaje C y Python permitirán la realización del mismo.

\subsection{Viabilidad económica}

Considerando que ya hay asistentes de programación gratuitos en internet nuestra única opción económica es poner en manos de los usuarios donaciones para el proyecto como ya hace Eclipse.

\subsection{Viabilidad legal}

No hay ninguna oposición legal a este proyecto siempre que se declare la autoría tanto del parser\footnote{Enlace al repositorio del parser original:\url{https://github.com/eliben/pycparser}} como de los iconos\footnote{Fuente de los iconos:\url{https://www.flaticon.com/packs/essential-set-2}} utilizados en el proyecto.
